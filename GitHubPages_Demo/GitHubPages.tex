\documentclass[a4paper,11pt]{article}

\usepackage{enumerate}
\usepackage[osf]{mathpazo}
\usepackage[T1]{fontenc}
\usepackage{setspace}
\pagenumbering{arabic}
\setlength{\parindent}{10ex}
\linespread{1.66} % All text should be double-spaced with occasional exceptions for tables. 
\raggedright


\begin{document}

\begin{flushright}
Version dated: \today
\end{flushright}
\begin{center}

%Title
\noindent{\Large{\bf{A quick GitHub pages tutorial}}}\\
\bigskip
%Author
\noindent{Thomas Guillerme - guillert@tcd.ie - http://tguillerme.github.io/}\\

\end{center}

This is a really quick tutorial on how to create GitHub pages, whether you want a specific \textbf{project page} or just a \textbf{user page} (e.g. stuff about you're life).
The idea is that GitHub allows to create a repository that is hosted as a website.
Therefore you can interact with this repository as usual (collaboration, version control, etc...) but it will appear as a web page instead of the classical repository visualisation of GitHub.
%Demo: Go through each step on screen (no submitting)
\section{Generating the page}
It's dead easy\footnote{Involves just clicking.}, just follow these steps (you might have already done the two first ones!):
\begin{enumerate}
    \item{Register on GitHub:}\\
    https://github.com/

    \item{\textit{Download a GUI client (recommended):}}\\
        \begin{itemize}
        \item{Windows: https://windows.github.com/}
        \item{Mac: https://mac.github.com/}
        \item{Linux/Nerd\footnote{If you're a bit more familiar with not only clicking but also copy/pasting stuff in the terminal window.}: You're grant, it's really easy on terminal too.}
        \end{itemize}

    \item{Generate your page:}\\
    https://pages.github.com/
\end{enumerate}

\section{Editing the page}
\setlength{\parindent}{5ex}
Ones every step is completed, let the magic happen and just modify the content depending on your mood straight from your computer. Then save/commit/push the modified content on GitHub and it will instantly update your on line repository.
%Demo: Submit a dummy tutorial page containing a link to this pdf.

GitHub pages are written in html language, if you're familiar with it, just have fun, if not here is a non-exhaustive list of resources:
\par
\begin{itemize}
    \item{Trial and error, get your hands dirty!}
    \item{http://www.w3schools.com/}
    \item{http://www.htmldog.com/guides/html/beginner/}
    \item{http://html.net/}
\end{itemize}
Note that you can create pages in any other web language too.
It's not much more complicated than that!

\newpage

\section{Some more details}
\subsection{User or project page?}
\setlength{\parindent}{5ex}
The user page will create a new repository on GitHub and your computer containing all the web pages information and resources.

The project page will link directly to one of your repositories, if you're familiar with GitHub, it's like a README file but in more fancy.
You can generate it automatically\footnote{Involves just clicking.} or manually\footnote{Also involves a bit of copy/pasting stuff in the terminal window - https://help.github.com/articles/creating-project-pages-manually
}.
\par

\subsection{What's in the project repository}

\begin{spacing}{1.1}
user.github.io/ \\
     \qquad  $\rightarrow$ index.html \\
     \qquad  $\rightarrow$ any other page.html \\
     \qquad  $\rightarrow$ stylesheet folder \\
     \quad \quad \qquad $\rightarrow$ .css file \\
     \qquad  $\rightarrow$ any other folder \\
     \quad \quad \qquad $\rightarrow$ some file \\
\end{spacing}

\subsection{Domain name?}
If you happen to have a domain name (e.g. checkmyawesomecv.com) there's a way to do it by adding a CNAME file to your repository.

\noindent{https://help.github.com/articles/setting-up-a-custom-domain-with-github-pages}

\end{document}
